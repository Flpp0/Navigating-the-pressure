\chapter*{Italian abstract}

Questo studio indaga la relazione tra le fluttuazioni del sodio plasmatico e la pressione intracranica (ICP) nei pazienti con trauma cranico (TBI), utilizzando algoritmi di machine learning per analizzare dati ad alta frequenza provenienti da una banca dati nazionale. I livelli di sodio plasmatico sono fondamentali per mantenere l’equilibrio osmotico e l’omeostasi cerebrale, e le deviazioni possono influire significativamente sull’ICP, un fattore chiave per i risultati clinici post-TBI.\\

Vengono esaminate le variazioni relative del sodio plasmatico (deltaNa) e dell’ICP (deltaICP) per valutare l’influenza della variabilità del sodio sulla ICP. Inoltre, è stato quantificato il tempo trascorso dai pazienti con valori di sodio e ICP al di sopra o al di sotto delle soglie stabilite — definiti come Na Dose e ICP Dose — per determinarne l’impatto sull'outcome. Sono state considerate le fluttuazioni del sodio sopra la soglia (ipernatremia), sotto la soglia (iponatremia) e all’interno dei limiti normali.

Il dataset utilizzato è una sotto-parte di una banca dati ad alta risoluzione, raccolta da una rete di unità di terapia intensiva (ICU) italiane, che comprende 63 ospedali e 77 ICU in tutto il paese. Questa banca dati contiene dati fisiologici ad alta intensità e frequenza, consentendo un monitoraggio preciso del sodio e dell’ICP nei pazienti critici con TBI. Gli algoritmi di machine learning sono stati applicati a questo dataset per rilevare pattern e relazioni che potrebbero non essere immediatamente evidenti con approcci statistici tradizionali.

I pazienti sono stati categorizzati in tre Livelli di Intensità Terapeutica (TIL) per valutare l’impatto degli interventi clinici sul controllo dell’ICP e stimare la compliance cerebrale utilizzando i TIL come marker surrogato. \\

I risultati dimostrano una forte correlazione tra deltaNa e deltaICP, con fluttuazioni del sodio che portano a una maggiore variabilità dell’ICP. Inoltre, periodi prolungati di Na Dose e ICP Dose elevati sono stati associati a esiti clinici peggiori, sottolineando la necessità di una gestione precisa dei livelli di sodio nei pazienti con TBI.