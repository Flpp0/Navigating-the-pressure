\chapter*{Abstract}

We investigate the relationship between plasma sodium fluctuations and intracranial pressure (ICP) in traumatic brain injured (TBI) patients, employing machine learning algorithms to analyze data from a national database. Plasma sodium levels are critical in maintaining osmotic balance and cerebral homeostasis, and deviations can significantly impact ICP, a key factor in patient outcomes post-TBI.\\

The study examines relative changes in plasma sodium (deltaNa) and ICP (deltaICP) to assess the influence of sodium variability on ICP dynamics. Additionally, the time patients spent with sodium and ICP values above or below established thresholds — termed Na Dose and ICP Dose — was quantified to determine the impact of prolonged dysregulation on patient outcomes. The study consider sodium fluctuations above threshold (hypernatremia), below threshold (hyponatremia) as well as within normal limits.

The dataset utilised is a subpart of a database sourced from a network of Italian intensive care units (ICUs), comprising 63 hospitals and 77 ICUs. This contains high-intensity and high-frequency data, enabling precise monitoring of both sodium and ICP in TBI patients. Machine learning algorithms were applied to uncover patterns and relationships that may not be immediately apparent through traditional statistical approaches.

Patients were categorized into three Therapy Intensity Levels (TIL) to evaluate the impact of clinical interventions on ICP control and to estimate cerebral compliance using TIL as a surrogate marker. \\

The results demonstrate a strong correlation between deltaNa and deltaICP, with significant sodium fluctuations leading to greater ICP instability. Moreover, prolonged periods of  Na Dose and ICP Dose were associated with poorer clinical outcomes, emphasizing the need for precise management of sodium levels and ICP in TBI care.