\chapter[Na\textsuperscript{\scriptsize+}vigating the pressure, sodium fluctuations]{Na\textsuperscript{\scriptsize+}vigating the pressure, sodium fluctuations}

\textit{The team behind this study started as a research team for a three-day datathon event in Lecco, Italy in November 2023 - the GiViTHON - endorsed by the Clinical Data Science of Istituto di Ricerche Farmacologiche Mario Negri IRCCS and Politecnico di Milano. Ten multidisciplinary teams composed of clinicians, nurses, statisticians, and biodata analysts have been challenged to respond to clinical questions and use the subset database for population selection and   key variables extractions.} 

\section {Introduction}
Traumatic brain injury (TBI) is a critical public health issue that frequently leads to long-term disabilities or death. After a TBI, the brain’s ability to regulate itself can be significantly compromised, resulting in a cascade of physiological disruptions. One of the most important of these is impaired autoregulation, which leads to instability in intracranial pressure (ICP) and compromised cerebral perfusion pressure (CPP). Consequently, cerebral blood flow can be reduced, increasing the risk of \textit{secondary brain injury} through ischemia, hypoxia, and further swelling.

The primary aim of treatment is to prevent secondary brain injury. This concept relies on distinguishing between injured brain tissue and salvageable brain tissue. While one portion of the brain may be irreversibly damaged, other areas remain viable but at risk due to inadequate blood flow or rising ICP. Failure to manage these can worsen the damage in these vulnerable regions, ultimately exacerbating the overall brain injury.\\

Dysnatriemias have already been recognized as a marker of severity of disease and related to mortality in the critically ill patient. There is however accumulating evidence that even mild abnormalities of serum sodium could be linked to disease severity and mortality, including changes within the normal range\cite{sakrFluctuationsSerumSodium2013}.

Serum sodium levels are tightly regulated in the human body, but patients with TBI are at higher risk of developing dysnatremias due to various factors such as hyperosmolar therapy, diabetes insipidus and SIADH. Both hypernatremia\cite{maggioreRelationIncidenceHypernatremia2009a}\cite{vedantamMorbidityMortalityAssociated2017a} and hyponatremia \cite{yumotoPrevalenceRiskFactors2015a} have been linked to higher mortality rates and worse outcomes in traumatic brain injury (TBI) patients as well as in aneurysmal subarachnoid hemorrhage (aSAH)\cite{labibSodiumItsImpact2024}\cite{balesEffectHyponatremiaSodium2016}, highlighting the importance of sodium regulation as a key therapeutic goal for individuals with elevated intracranial pressure. While managing serum sodium levels is important, fluctuations in sodium concentration, rather than just the absolute levels, may also  contribute to injury.\\

We therefore investigate the relationship between fluctuations in plasma sodium (deltaNa) and intracranial pressure (deltaICP) to evaluate how sodium variability influences ICP dynamics. Fluctuations are represented as time patients spent with sodium and ICP levels above or below predefined thresholds — referred to as Na Dose and ICP Dose — as well as  fluctuations within physiologic range.\\

\section {Materials and methods}
\subsection{Study population}
The cohort of critically ill patients with TBI was enrolled from a subpart of the Margherita3 electronic health record database - developed by the Italian Group for the Evaluation of Interventions in Intensive Care Medicine\cite{finazziDataCollectionResearch2018} - which contains the comprehensive clinical data of 5730\textcolor{red}{CONTROLLARE NUMERO PAZIENTI} patients of about 70 italian intensive care units (ICUs) between \textcolor{red}{AGGIUNGERE LE DATE}. 

The patients were searched based on the International Classification of Diseases (ICD-10) code. The inclusion criteria identified 411 \textcolor{red}{CONTROLLARE NUMERO PAZIENTI} patients who were diagnosed with TBI.\\

\textcolor{red}{FIL CONTROLLIAMOLI ASSIEME}

\textcolor{blue}{4 e 5 di inclusion e 4 e 5 di esclusion sono la stessa cosa. Per il discorso exclusion abbiamo rimosso pazienti sia in quanto non avevano una numerosità sufficiente per interpolazione sia se non avevano una intersezione tra le due curve (sodio e IPC) di almeno 48h. Per admission criteria si potrebbe mettere direttamente il numero di valori Na e ICP senza mettere che ne abbiano almeno uno e poi mettere anche almeno x valori.
Discorso intersezione e valori minimi è per la parte di doseNa dosePIC e deltadelta.}

The inclusion criteria were as follows: (1) age $\geq$ 18 years, (2) TBI as cause of admission, (3) at least an intracranial pressure (ICP) reading, (4) at least a plasma sodium value.
The exclusion criteria were as follows: (1) expected ICU admission patients, (2) ICU admission after elective surgery, (3) ICU stay of less than 72h (20 patients), (4) <1 ICP measurement available , (5) <1 sodium measurement. Patients who died within 72 hours of ICU admission were excluded to focus the analysis on those with potentially salvageable brain injuries.

\subsection{Data Collection}
\textcolor{red}{FIL CONTROLLIAMOLI ASSIEME}

\textcolor{blue}{Non so nemmeno se valga la pena scrivere di Python e SQL ahah sincero :)}
Baseline parameters within the first day after ICU admission were collected using Python as  Structure Query Language (SQL) parser, including demographics (e.g., sex, age, weight, admission type), comorbidities (e.g., myocardial infarction, congestive heart failure, chronic pulmonary disease, and renal disease), assessment scale scores (Glasgow Coma Scale, GCS and Acute Physiologic Assessment and Chronic Health Evaluation, APACHE II Scoring System), \textcolor{blue}{DAVVERIO ABBIAMO MESSO ANCHE QUESTI (vital signs)??} vital signs (e.g., heart rate, respiratory rate, blood pressure, intracranial pressure and pupil reactivity). 
We gathered all serum sodium measurements from ICU admission up to 14 days, or until death, whichever occurred first. Additionally, we recorded data on various therapies, including extraventricular drainage, use of osmotherapy, barbiturate coma, hypothermia, among others. Lastly, we collected outcome data such as 14-day ICU mortality and overall ICU mortality.

All the features extracted can be seen in Tables \ref{tab:categorical_features} and \ref{tab:continuous_features}.
\begin{table}[h!]
	\centering
	\tiny % Reduce font size
	\begin{tabular}{lll}
		\hline
		\textbf{Feature} & \textbf{Category} & \textbf{Count (\%)} \\ \hline
		SESSO & M & 321 (78.10\%) \\
		SESSO & F & 90 (21.90\%) \\
		TIPO & Chirurgico d’urgenza & 229 (55.72\%) \\
		TIPO & Medico & 182 (44.28\%) \\
		OUTCOME & non deceduto & 357 (86.86\%) \\
		OUTCOME & deceduto & 54 (13.14\%) \\
		ICPm & True & 372 (90.51\%) \\
		ICPm & False & 39 (9.49\%) \\
		Hypernatremia\_within\_7\_days & True & 259 (63.02\%) \\
		Hypernatremia\_within\_7\_days & False & 152 (36.98\%) \\
		Hyponatremia\_within\_7\_days & False & 257 (62.53\%) \\
		Hyponatremia\_within\_7\_days & True & 154 (37.47\%) \\
		\hline
	\end{tabular}
	\caption{Categorical Feature Statistics}
	\label{tab:categorical_features}
\end{table}

\begin{table}[h!]
	\centering
	\tiny % Reduce font size
	\begin{tabular}{lccc}
		\hline
		\textbf{Feature} & \textbf{Mean (unit)} & \textbf{Standard Deviation (unit)} & \textbf{Range (unit)} \\ \hline
		Altezza & 173.24 (cm) & 12.61 (cm) & [160.63 - 185.86] (cm) \\
		Peso & 81.53 (kg) & 14.76 (kg) & [66.77 - 96.28] (kg) \\
		BMI & 25.46 (kg/m$^2$) & 4.15 (kg/m$^2$) & [21.31 - 29.61] (kg/m$^2$) \\
		Best\_GCS\_Total\_Response & 3.15 & 4.03 & [-0.88 - 7.18] \\
		Worst\_GCS\_Total\_Response & 2.41 & 3.14 & [-0.74 - 5.55] \\
		Best\_GCS\_Motor\_Response & 2.56 & 2.09 & [0.47 - 4.65] \\
		Worst\_GCS\_Motor\_Response & 1.43 & 1.44 & [-0.01 - 2.88] \\
		Fixed\_Pupil\_Flag & 0.06 & 1.00 & [-0.94 - 1.06] \\
		AGE\_CLEANED & 54.83 (years) & 18.08 (years) & [36.75 - 72.90] (years) \\
		APACHE\_II\_Score & 27.46 & 4.83 & [22.64 - 32.29] \\
		First\_Sodium\_Value & 139.59 (mmol/L) & 4.48 (mmol/L) & [135.11 - 144.07] (mmol/L) \\
		Time\_to\_Hypernatremia & 9307.56 (minutes) & 8572.06 (minutes) & [735.50 - 17879.62] (minutes) \\
		Time\_to\_Hyponatremia & 13849.85 (minutes) & 8421.55 (minutes) & [5428.30 - 22271.41] (minutes) \\
		Na\_min\_first\_7\_days & 135.12 (mmol/L) & 8.76 (mmol/L) & [126.36 - 143.88] (mmol/L) \\
		Na\_max\_first\_7\_days & 148.33 (mmol/L) & 6.76 (mmol/L) & [141.57 - 155.09] (mmol/L) \\
		Na\_SD\_first\_7\_days & 3.53 (mmol/L) & 2.44 (mmol/L) & [1.08 - 5.97] (mmol/L) \\
		\hline
	\end{tabular}
	\caption{Continuous Feature Statistics}
	\label{tab:continuous_features}
\end{table}



\subsection{Generation of Variables and Definitions}

\paragraph{Serum Sodium Concentration on Admission and Na Dose}

Serum sodium concentration on admission was defined as the first available sodium measurement after admission to the Intensive Care Unit (ICU). Hyponatremia and hypernatremia were defined as conditions where at least three consecutive serum sodium values were below 135~mmol/L or above 145~mmol/L, respectively. The \textit{Na Dose} was defined as the area under the curve (AUC) where plasma sodium remains below (for hyponatremia) or above (for hypernatremia) the respective thresholds over time. This metric quantifies the extent and duration of sodium abnormalities, providing a cumulative measure of sodium imbalance.

\paragraph{Glasgow Coma Scale (GCS)}

The Glasgow Coma Scale (GCS) on admission was defined as the first available GCS score after admission to the ICU, with a specific focus on the motor component (GCSm). Additionally, we evaluated the best and worst GCS scores within the first 24~hours to assess the patient's neurological status over time. 

\paragraph{Intracranial Pressure (ICP) Dose}
The \textit{ICP Dose} is defined as the area under the curve (AUC) where intracranial pressure (ICP) exceeds 22 mmHg, a threshold linked to worse outcomes in traumatic brain injury. It quantifies the combined magnitude and duration of elevated ICP, offering a measure of intracranial hypertension severity.


\paragraph{Analysis of Plasma Sodium and ICP Fluctuations}

\textcolor{blue}{Mi serve un modo per introdurre il discore del delta delta altrimenti si fa fatica a presentare la procedura che abbiamo implementato}.
The aim of this analysis is to validate that the ratio of changes in intracranial pressure (ICP) to changes in plasma sodium (Na), denoted as \(\frac{\Delta \text{ICP}}{\Delta \text{Na}}\), contains critical information that can predict patient outcomes. By demonstrating the relevance of this metric, we highlight its potential importance for clinical decision-making in the ICU, where timely and accurate adjustments to treatment plans can significantly impact patient recovery.

\textcolor{blue}{Da qui si comincia}
\textcolor{blue}{DEVO INTRODURRE IL CONCETTO DEL DELTA che è diverso dal concetto di tau. Dobbiamo parlare un attimo di questo. }
To achieve this, we first generated continuous curves for both Na and ICP using linear interpolation. For Na, linear interpolation was applied between the first and last available sodium concentration values, generating interpolated values at one-minute intervals across the entire timeframe. The same interpolation process was applied to the ICP data, resulting in a continuous curve of ICP values at one-minute intervals. This step ensures a consistent temporal resolution for both variables, enabling a detailed analysis of their fluctuations.

Next, we focused on the overlapping time periods where both Na and ICP curves had data points available. To accurately capture the temporal relationship between changes in Na and subsequent changes in ICP, it is essential to align these time-series data correctly. The concept of $\tau$ is crucial in this context: it represents the time delay between a change in Na (cause) and its resulting effect on ICP (effect). For instance, if we consider time points $T_0$ and $T_1$ for Na, and the same time points for ICP, it would be incorrect to directly correlate these changes without accounting for the time it takes for a change in Na to affect ICP. Therefore, we need to align the curves by shifting the ICP curve backward by $\tau$ minutes relative to the Na curve.

\textcolor{blue}{Da aggiungere che questa procedura ha senso perché clinicamente è accettato che Na influenza PIC e in più abbiamo validato questa con il metodo della Granger Causality: nel 60\% dei casi risultato statisticamente valido. Bisogna scriverlo meglio secondo me.}

To determine the optimal $\tau$, we initially employed Granger causality analysis, a statistical method that assesses whether past values of one time series can predict future values of another. This step served to confirm the well-established directional relationship where changes in Na levels influence subsequent changes in ICP (Na $\rightarrow$ ICP) as documented in the literature. However, while Granger causality validated the direction of influence, it \textit{did not provide a statistically significant universal value for} $\tau$ across the patient population. This finding indicated that a fixed $\tau$ could not adequately describe the relationship for all patients.

Given this limitation, we adopted an empirical approach using a grid search to identify the optimal $\tau$ for each patient. The grid search involved testing various $\tau$ values within a plausible range based on clinical knowledge and previous studies \textcolor{blue}{Ti va se mettiamo proprio i valori?} For each $\tau$ value tested, the Na and ICP curves were aligned accordingly, and we evaluated which alignment produced a dataset that resulted in the best performance of downstream machine learning (ML) models. This data-driven method allowed us to select the most appropriate $\tau$ for each patient, taking into account individual variability in the time delay between changes in Na and their effects on ICP.

By selecting the $\tau$ that maximized the predictive accuracy of the ML models, we ensured that the time-series alignment was optimized for subsequent analysis. This iterative optimization provided a tailored approach for each patient, enhancing the reliability of the calculated \(\frac{\Delta \text{ICP}}{\Delta \text{Na}}\) values and their use in predicting patient outcomes.

In summary, the procedure involved generating interpolated time-series data for Na and ICP, confirming the Na $\rightarrow$ ICP relationship with Granger causality, and using a grid search approach to empirically determine the optimal $\tau$ for aligning these curves. By validating that \(\frac{\Delta \text{ICP}}{\Delta \text{Na}}\) contains relevant and essential information for predicting outcomes, we emphasize its potential value for informing ICU treatment decisions.
\\
\textcolor{blue}{QUESTA UN' ALTRA IDEA}
\paragraph{Analysis of Plasma Sodium and ICP Fluctuations NEW}

In this analysis, we explore the predictive power of changes in intracranial pressure (ICP) relative to changes in plasma sodium (Na), denoted as $\frac{\Delta \text{ICP}}{\Delta \text{Na}}$, as well as the dynamic features extracted directly from the Na and ICP signals themselves. The aim is to demonstrate that both the delta-delta relationship and the dynamic features of these signals—specifically those that capture relative changes over time—contain valuable information that can predict patient outcomes in the ICU. This multi-faceted approach allows us to consider not only the interactions between Na and ICP but also their standalone behaviors over time, focusing on how these parameters change relative to their baselines rather than on their absolute values.

To achieve this, the analysis is divided into two main parts:

\paragraph{Feature Extraction from Na and ICP Signals:}

The first part of the analysis focuses on the individual Na and ICP signals to extract their dynamic characteristics over time. For each patient, we generated continuous curves for both Na and ICP using linear interpolation. The interpolated data points were generated at one-minute intervals across the entire observation period, providing a high-resolution temporal analysis of the signals.

From these interpolated signals, we extracted several features that characterize the temporal behavior of Na and ICP independently. These features include dynamic metrics such as the range, variability, skewness, kurtosis, and spectral entropy, which capture the fluctuation and variability of the signals over time. We also calculated the "dose" of Na and ICP, defined as the area under the curve (AUC) where the respective signal values exceed or fall below clinically significant thresholds (e.g., hypernatremia threshold for Na and intracranial hypertension threshold for ICP). These dose features are particularly useful as they represent cumulative deviations from normal levels, emphasizing the duration and extent of abnormal states rather than just their occurrence.

Importantly, we focused on extracting \textit{relative value features} rather than absolute values. For instance, while absolute metrics like mean or standard deviation were initially considered, we found that they often correlated with dynamic features such as "dose." Since our primary goal was to understand the temporal dynamics and the evolving nature of these signals, we prioritized features that describe changes and relative behaviors over time. This focus ensures that the analysis emphasizes how signals deviate from their baselines and interact dynamically, which is more relevant for predicting patient outcomes.

\paragraph{Delta-Delta Analysis of Na and ICP Signals:}

The second part of the analysis focuses on the relationship between changes in Na and changes in ICP using the delta-delta approach. This involves examining the ratio of changes in ICP to changes in Na ($\frac{\Delta \text{ICP}}{\Delta \text{Na}}$) and identifying how this ratio varies over time for each patient. To align these time-series data correctly and capture the temporal lag between a change in Na (cause) and its resulting effect on ICP (effect), we introduced the concept of $\tau$.

Using a combination of Granger causality analysis and a grid search approach, we determined the optimal $\tau$ for each patient to align the Na and ICP signals effectively. Once aligned, we calculated the delta-delta series for overlapping time periods, where both signals had valid data points. From these series, we extracted features such as mean, median, skewness, kurtosis, and entropy, focusing on how these ratios fluctuate and change over time. These delta-delta features further emphasize the importance of relative dynamics between Na and ICP, aligning with our goal of understanding signal behavior through their changes rather than absolute values.

\paragraph{Integrating Both Analyses for Predictive Modeling}

By combining the features extracted from both the Na and ICP signals themselves and their delta-delta interactions, we provide a comprehensive dataset that captures both direct signal behaviors and their dynamic relationships. However, for the purpose of training machine learning models, we focused primarily on \textit{relative value features}. This decision was motivated by our interest in capturing the dynamics and changes in signals rather than static levels.

While absolute features like the mean or standard deviation of the signals were initially extracted, they were excluded from the final model because they were often redundant when compared with more dynamic metrics, such as the "dose" of Na or ICP. Our primary focus was on understanding how signals evolve and interact over time, which is best captured through relative and dynamic features. This approach helps to reduce redundancy in the feature set and ensures that the models are trained on the most informative aspects of the data, emphasizing the patterns and interactions that are more likely to provide actionable insights for clinical decision-making.

\paragraph{Conclusion}

This dual analysis approach underscores the importance of considering both the individual characteristics of Na and ICP signals and their interactions. By focusing on relative value features in our predictive models, we highlight the dynamics of signal interactions, providing a more nuanced understanding of patient states. The delta-delta analysis reveals insights into the relationship between Na and ICP fluctuations, while the direct signal features capture the dynamics of these physiological parameters over time. Together, they offer a holistic view of patient dynamics in the ICU, supporting more informed clinical decision-making.


\subsection{Statistical analysis and machine learning models development}
\textcolor{red}{FIL LASCEREI SCRIVERE A TE}

\section {Results}
\subsection{Patient Demographics and Clinical Features}
\textcolor{red}{INSERIRE TABELLA DESCRITTIVE}

\subsection{Relationship between serum sodium fluctuations and intracranial pressure}
\textcolor{red}{VALUTARE INSIEME SE E COME HA SENSO INSERIRE QUESTO PARAGRAFO, IN RIFERIMENTO A QUEL 60\% CIRCA}

\subsection{Relationship Between Na Dose and ICP Dose on outcome}
\textcolor{red}{SAREBBE IL MODELLO 2}

\subsection{Relationship between sodium fluctuations and ICP fluctuations (DeltaDelta) on outcome}
\textcolor{red}{SAREBBE IL MODELLO 3}

\subsection{Subgroup analysis: TIL}
\textcolor{red}{SAREBBE IL DISCORSO SUI TIL CON LE VARIE TABELLE}

\section{Discussion}

\section{Conclusion}





